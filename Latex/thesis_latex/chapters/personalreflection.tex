\chapter{Personal Reflection}\label{sec:personalreflection}

Personally, this Bachelor thesis research has been instructive and beneficial in many different ways. Most importantly, I have had the opportunity to dedicate my entire time to learn about the subject of quantum information and, specifically, quantum machine learning in detail. Since these subjects are not taught within the curriculum of the Maastricht Science Programme it was especially amazing to challenge myself to these notoriously difficult subjects within the intersection of quantum physics and computer science. In doing so, I have learned more about the methods of theoretical physics and gained more experience in scientific programming with Octave, Python and F\#. Without prior knowledge of F\#, I was able to learn how to simulate quantum computations and quantum machine learning algorithms using the quantum simulation toolsuite Liqui$\ket{}$. Furthermore, I was able to use the first cloud-based quantum computer, the so-called IBM Quantum Experience, publicly released in May 2016 by IBM.


%The Centre for Quantum Technology at the University of KwaZulu-Natal in South Africa has been a wonderful host for my Bachelor thesis research. My supervisor Prof. Francesco Petruccione has become a mentor and friend and provided me with great opportunities for personal and academic growth. Furthermore, my collaborator Maria Schuld has shared vasts amount of knowledge, tricks and ideas with me and has always been a great help during my research work. Alongside my research, I had the possibility to attend many great lectures, seminars and three conferences which were all generously funded by the Centre for Quantum Technology. This enabled me to get to know many renowned scientists working in the fields of quantum information, quantum machine learning, open quantum systems, quantum optics and quantum cryptography. Attending the Quantum Machine Learning Workshop at the Dolphin Coast in July 2016 provided me with a broad overview of the field of quantum machine learning and was the first time that I ever attended a scientific conference. In November 2016, I was provided with the opportunity to present my research at the 4\textsuperscript{th} South African conference for Quantum Information Processing, Communication and Control in Cape Town. This opened the possibility for a collaboration with an experimental research group in Israel working on quantum computation with trapped ions. In January 2017, I am invited to the NiTheP Chris Engelbrecht Quantum Machine Learning Summer School where I will give three workshop sessions on quantum machine learning using the Liqui$\ket{}$ framework and the IBM Quantum Experience.