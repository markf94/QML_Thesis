\chapter{Conclusion}\label{sec:conclusion}

Quantum-enhanced machine learning aims to harness the properties of quantum mechanical systems to enhance the performance of classical machine learning algorithms beyond the reach of classical computers. This becomes increasingly important since 'big data' pushes computational resources and classical algorithms to their limits. However, current quantum computing efforts have not yet achieved a large-scale universal quantum computer and, thus, most of the current quantum-enhanced machine learning research is purely theoretical. Yet, small-scale quantum computers have been realised already and up to 30 qubits can still be simulated on a conventional laptop. Thus, this research is part of the attempt to shift quantum-enhanced machine learning into a more applied field by establishing proof-of-principle simulations of two variants of a distance-weighted quantum $k$-nearest neighbour (kNN) algorithm using three small-scale supervised machine learning tasks.
%On the other hand, IBM has enabled public access to their five-qubit quantum computer and small quantum computers with up to 30 qubits can still be simulated on conventional laptops. 

Firstly, the qubit-based kNN algorithm by \citeA{Schuld2014} was combined with a quantum state preparation routine by \citeA{Trugenberger2001} to classify various 9-bit RGB colours into the classes \emph{red} and \emph{blue}. Microsoft's software architecture Liqui$\ket{}$ was used to simulate the algorithm on two levels of difficulty. In the easy evaluation stage, the quantum kNN achieved a classification accuracy of 75\% which was subsequently raised to 100\% in the more challenging evaluation stage with a larger training dataset. These results show that the qubit-based kNN routine is a very effective classification algorithm concerning 9-bit RGB colours.

Aiming for an implementation with the IBM Quantum Experience, an amplitude-based kNN algorithm (aKNN) was developed by \citeA{SchuldFingerhuth}.  A simple binary Bloch vector classification tasks was considered to keep the requirements on the quantum hardware as small as possible,. The necessary quantum compiling steps mapping the quantum state preparation routine and the aKNN algorithm to the IBM quantum hardware were outlined. For that particular classification task, it was shown that it could not be implemented within IBM's 40 quantum gate slots. Even though an implementation might be feasible using different datasets, the research demonstrated that there are cases which require relatively large gate sequences for implementation. As an alternative, the Bloch vector classification problem using the aKNN algorithm was simulated in Liqui$\ket{}$. The simulated algorithm led to the correct classification of two Bloch vectors. Lastly, the aKNN algorithm was simulated for the classification of discrete Gaussian distributions constituting a higher-dimensional classification problem using a slightly more advanced quantum state preparation routine. The simulated algorithm classified 83.3\% of various discrete Gaussian distributions correctly.

Overall, the Liqui$\ket{}$ quantum simulations clearly demonstrated the effectiveness of both quantum kNN algorithms. When combined with more general quantum state preparation routines, these algorithms bear the potential to speed up the processing of 'big data'. Thus, a breakthrough in the development of a universal quantum computer would not just only constitute a major milestone in physics but most probably also revolutionise the field of machine learning.
