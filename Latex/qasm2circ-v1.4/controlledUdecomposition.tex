%
% File:   controlledUdecomposition.qasm
% Date:   22-Mar-04
% Author: Mark Fingerhuth
% 
% Two-qubit controlled-U gate decomposition
% 
%   def C,0,'C'
%   def B,0,'B'
%   def A,0,'A'
%   def Ryt,0,'\m{1&0\cr0&e^{i\alpha}}'
% 
% 	qubit	q0
%   qubit q1
% 
%   C q1
%   cnot q0,q1
%   B q1
%   cnot q0,q1
%   A q1
%   Ryt q0

%  Time 01:
%    Gate 00 C(q1)
%  Time 02:
%    Gate 01 cnot(q0,q1)
%  Time 03:
%    Gate 02 B(q1)
%  Time 04:
%    Gate 03 cnot(q0,q1)
%  Time 05:
%    Gate 04 A(q1)
%    Gate 05 Ryt(q0)

% Qubit circuit matrix:
%
% q0: n  , gBxA, n  , gDxA, gExA, n  
% q1: gAxB, gBxB, gCxB, gDxB, gExB, n  

\documentclass[11pt]{article}
\input{xyqcirc.tex}

% definitions for the circuit elements

\def\gAxB{\op{C}\w\A{gAxB}}
\def\gBxA{\b\w\A{gBxA}}
\def\gBxB{\o\w\A{gBxB}}
\def\gCxB{\op{B}\w\A{gCxB}}
\def\gDxA{\b\w\A{gDxA}}
\def\gDxB{\o\w\A{gDxB}}
\def\gExB{\op{A}\w\A{gExB}}
\def\gExA{\op{\m{1&0\cr0&e^{i\alpha}}}\w\A{gExA}}

% definitions for bit labels and initial states

\def\bA{ \q{q_{0}}}
\def\bB{ \q{q_{1}}}

% The quantum circuit as an xymatrix

\xymatrix@R=5pt@C=10pt{
    \bA & \n   &\gBxA &\n   &\gDxA &\gExA &\n  
\\  \bB & \gAxB &\gBxB &\gCxB &\gDxB &\gExB &\n  
%
% Vertical lines and other post-xymatrix latex
%
\ar@{-}"gBxB";"gBxA"
\ar@{-}"gDxB";"gDxA"
}

\end{document}
